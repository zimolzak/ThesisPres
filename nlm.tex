\documentclass[12pt]{report}
\usepackage{speech}
\begin{document}
\begin{large}

Thanks everyone for coming to my talk. Medication adherence is defined
generally as ``the extent to which patients take medications as
prescribed.'' First I'll cover four facts about adherence to put my
research in context. %title

THE FIRST FACT IS: Adherence is good. %fact

Poor adherence is associated with increased mortality and morbidity in
clinical trials, and with increased hospitalizations in observational
studies. 30 to 70 \% of adverse drug event-related hospitalizations
are attributable to poor adherence, and this increases costs. One
study estimates 289 billion dollars annually in drug-related morbidity
(including nonadherence) in the US. %nehi

Furthermore, the Centers for Medicare and Medicaid Services has tied
adherence for certain drugs to Medicare bonus payments. CMS measures
adherence using prescription fills. Therefore, if you're a Medicare
plan manager, better population adherence gets you better
reimbursement. This CMS definition informed how we operationalized the
definition of adherence. %cms

SECOND FACT: Nonadherence is common. %fact

These are two studies of people who have been newly prescribed
statins. By 6 months, half have stopped taking statins. This upper
curve is for those who have just recently had a heart attack.
Nonadherence is still common, but less so. %jackevicius

THIRD FACT: Adherence interventions work but can be complex. %fact

In one study, a pharmacist was trained by several other pharmacists, a
cardiologist, a geriatrician, a behavioral scientist, and a cognitive
psychologist, and then this expertly-trained pharmacist delivered an
intervention to patients over 9 months, using a tailored protocol. One
systematic review concludes, ``many of the adherence interventions for
long-term medications were exceedingly complex and labor-intensive. It
is therefore difficult to see how they could be carried out in
non-research settings.''

Some interventions are simpler. In this study, reduced copayment
improved adherence by 4--6 percentage points and led to small
improvements in clinical endpoints. Reducing copayment is arguably
simple compared to training a small army of multidisciplinary teams.
%nejm

In any case, complex interventions illustrate why we want to predict
adherence. We think the resource-intensive intervention is not for
everyone---some people will do well on their own.

FOURTH FACT: Some predictors of adherence don't work. %fact

One study tried to predict statin adherence from 19 features of the
patient, physician, and payment amounts. The resulting model had an
area under the ROC curve of 0.63, which means it didn't work so well.
Several papers state that we probably can't predict adherence from
someone's appearance, problem list, or physician or payment
characteristics. %chan

However, a study from my lab showed good model performance by using
past prescription claims to predict future prescription claims. But
this study didn't compare early detection to other predictors, and it
also selected a cohort with high mean adherence. %magda

This sets up my research, which is meant to build an early detection
model, but do 2 additional things. First, compare the early detection
predictor to traditional predictors. Second, use the CMS definition so
the model becomes relevant to pay-for-performance.

Are there any questions before I move on to research?








All right. Our data source was prescription claims from Aetna, on
600,000 commercial members, who met criteria for hyperlipidemia. %flow1

The major predictor of interest is statin adherence 90 days after
prescribing. We also calculated 15 traditional predictors including
number of pills, copayment, type of pharmacy, and heart attack 30 days
before statin initiation. We defined the dependent variable per CMS.
It is binary statin adherence: specifically whether Proportion of Days
Covered is greater or less than 80\%, for statins. It covers days
91--365. %timeline!!!

We excluded a few conditions: mainly anyone who is not newly
initiating a statin. This leaves 210,000 people. We trained logistic
regression on a random 2/3 set and tested on 1/3. %flow2

Here are simple characteristics of our cohort. They have few
medications, few recent heart attacks, and nonadherence is common.
%table1

This is the ROC curve for our multivariable model. Area under the
curve is 0.8, so we think this is a good model. I would still agree
that you can't make accurate predictions from demographic and
comorbidity data, but early monitoring is probably acting as a proxy
for characteristics that we can't measure in claims. This model has
positive predictive value 88\%, negative predictive value 53\%,
Sensitivity 69\%, Specificity 78\%. %roc!!!

Here are multivariable odds ratios. Associates of poor adherence fall
on the right; associates of good adherence are on the left. The
strongest predictor is low adherence in the early monitoring period,
with an odds ratio of 25. The point is not these runners up, but how
much the early detection predictor stands out. %odds

Finally, this is the really interesting part. 90 days of early
detection made clinical sense, but maybe we shouldn't use 90. This
shows model performance on the Y axis versus duration of early
monitoring on the X axis. The top curve is the cohort with 30 day
fills, and the bottom curve is those with 90 days. In the top curve, 0
to 30 days of monitoring data yields a weak model. If you wait for 31
days of data, you get a a better model, and if you wait til 40 you get
even better. Then you increase more slowly, for as long as you're
willing to wait. There is a trade-off between an early, less accurate
prediction; and a later more accurate one. %curves

Who could use this model? Firstly, insurance plan managers. Find
people filling a medicine for the first time, monitor their refills,
intervene early, and improve your pay-for-performance. Secondly,
clinics or ACOs could use this: similarly intervene early but with the
goal of improving their disease complication rates. %discussion

Now I want to discuss opportunities for future research. First\-ly
there are ways to measure adherence other than fills, which would add
information to this model.

Secondly, it's arbitrary to define good adherence as Proportion of
Days Covered greater than 80\%. I don't know exactly \emph{how much}
of your statin you need to take, and I haven't seen a paper using
clinical outcomes to answer this question.

Third, we're looking just at statin adherence predicting statin
adherence. What does antidepressant adherence have to say about statin
adherence, or vice versa? You can imagine a complex network of
possible correlations.

To sum up, demographic and payment characteristics are not good
predictors of statin prescription filling, but early statin adherence
is one useful predictor found in claims data. Medication adherence is
an important problem, and there are many important questions that we
will be able to answer about it in the near future.

I would like to thank the following people who made this research
possible. %thanks

% 10 min 49 sec.

\end{large}
\end{document}
% LocalWords:  operationalized ly
